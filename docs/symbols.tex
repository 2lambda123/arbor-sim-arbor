\begin{table*}[htp!]
    \begin{center}

    \begin{tabular}{llll}
        \hline
        quality & symbol & unit  & notes \\
        \hline
        energy     & $J$ & joule   $j$  & work to push 1 $N$ through 1 $m$ \\
        charge     & $q$ & coulomb $C$  & $6.25\cdot10^{18}$ electrons, $[A\cdot s]$ \\
        current    & $I$ & ampere  $A$  & $[C\cdot s^{-1}]$, $A$ is SI base unit\\
        voltage    & $V$ & volt    $V$  & potential work per unit charge \\
        resistance & $R$ & ohm $\Omega$ & recall Ohm's law $V=IR$ \\
        capacitance& $C$ & farad   $F$  & $C=\frac{q}{V}$, $[J\cdot C^{2}]$\\
        conductance& $g$ & siemens $S$  & $g=1/R$ i.e. inverse of resistance \\
        \hline
    \end{tabular}

    \begin{tabular}{llll}
        \hline
        symbol & unit & equivalents & SI base \\
        \hline
        $J$    & $j$      &  $J\cdot s^{-1}$, $V\cdot A$ &
            $kg\cdot m^{2}\cdot s^{-2}$ \\

        $q$    & $C$      & $s\cdot A$ &
            $s\cdot A$ \\

        $I$    & $A$  & $C\cdot s^{-1}$ &
            $A$ \\

        $V$    & $V$      & $W\cdot A$ &
            $kg\cdot m^{2}\cdot s^{-3}\cdot A^{-1}$ \\

        $R$    & $\Omega$ & $V\cdot A^{-1}$ &
            $kg\cdot m^{2}\cdot s^{-3}\cdot A^{-2}$ \\

        $C$    & $F$      & $C\cdot V^{-1}$  &
            $kg^{-1}\cdot m^{-2}\cdot s^{4}\cdot A^{2}$ \\
        $g$    & $S$      & $A\cdot V^{-1}$  &
            $kg^{-1}\cdot m^{-2}\cdot s^3\cdot A^2$ \\
        \hline
    \end{tabular}

    \end{center}
    \caption{Symbols and quantities.}
\end{table*}
%-------------------------------------------------------------------------------
\subsubsection{Units}
%-------------------------------------------------------------------------------
Reffering to the cable equation first defined in~\eq{eq:cable}
\begin{equation*}
    c_m \pder{V}{t} = \frac{1}{2\pi a r_{L}} \pder{}{x} \left( a^2 \pder{V}{x} \right) - i_m + i_e,
\end{equation*}

If the units are taken to be
\begin{itemize}
    \item $c_m = F\cdot cm^{-2}$
    \item $V = V$
    \item $a = cm$
    \item $r_L = \Omega\cdot cm$
\end{itemize}
Then the units of each term in equation are $A\cdot cm^{-2}$.
In practice, the units above are not used, for example distances are usually measured in $\mu m$ and areas in $cm^2$.
But if we work term-by-term, scaling for these factors is manageable.

A useful identity to use when performing the dimensional analysis relates capacitance and resistance
\begin{equation*}
    1~F = 1~\Omega^{-1} \cdot s
\end{equation*}

