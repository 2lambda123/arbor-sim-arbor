\section*{Introduction}

The cable equation is a nonlinear parabolic PDE that can be written in the form
\begin{equation}
    \label{eq:cable}
    c_m \pder{V}{t} = \frac{1}{2\pi a r_{L}} \pder{}{x} \left( a^2 \pder{V}{x} \right) - i_m + i_e,
\end{equation}
where
\begin{itemize}
    \item $V$ is the potential relative to the ECM $[mV]$
    \item $a$ is the cable radius $(mm)$, and can vary with $x$
    \item $c_m$ is the {specific membrane capacitance}, approximately the same for all neurons $\approx 10~nF/mm^2$. Related to \emph{membrane capacitance} $C_m$ by the relationship $C_m=c_{m}A$, where $A$ is the surface area of the cell.
    \item $i_m$ is the membrane current $[A]$. The total contribution from ion and synaptic channels is expressed as a the product of current per unit area $i_m$ and the surface area.
    \item $i_e$ is the electrode current flowing into the cell, divided by surface area, i.e. $i_e=I_e/A$.
    \item $r_L$ is intracellular resistivity, typical value $1~k\Omega$
\end{itemize}

Note that the standard convention is followed, whereby membrane and synapse currents ($i_m$) are positive when outward, and electrod currents ($i_e$) are positive inward.

The PDE in (\ref{eq:cable}) is derived from the following mass balance expression for a cable segment:
\begin{align}
    \int_{\Omega}{c_m \pder{V}{t} } \deriv{v} =
        & - \int_{\Gamma_{\text{left}}} \left( \frac{1}{r_L}\pder{V}{x} \right) \deriv{s}
          + \int_{\Gamma_{\text{right}}} \left( \frac{1}{r_L}\pder{V}{x} \right) \deriv{s} \nonumber \\
        & - \int_{\Gamma_{ext}} {(i_m - i_e)} \deriv{s}
    \label{eq:cable_balance}
\end{align}
where $\int_\Omega \cdot \deriv{v}$ is shorthand for the volume  integral over the segment $\Omega$, and $\int_\Gamma \cdot \deriv{s}$ is shorthand for the surface integral over the surface $\Gamma$.
The surface of the cable segment is sub-divided into the left, right and external parts of the surface.

The external surface $\Gamma_{ext}$ is the cell membrane, at the interface between the extra-cellular and intra-cellular regions.
The current, which is the conserved quantity in our conservation law, over the surface is composed of the synapse and ion channel contributions.
This is derived from a thin film approximation to the cell membrane, whereby the membrane is treated as an infinitesimally thin interface between the intra and extra cellular regions.

The left and right surface are the interface between the cable segment and its neighbour.

\subsection{Assumptions of the cable equation}
See \cite{lindsay_2004} for a detailed derivation of the cable equation, and extensions to the one-dimensional model that account for radial variation of potential.

The formulation in equations~\eq{eq:cable} and~\eq{eq:cable_balance} is based on the following expression in three dimensions (based on Maxwell's equations adapted for neurological modelling)
\begin{equation}
    \nabla \cdot \vv{J} = 0,
\end{equation}
where $\vv{J}$ is current density (units $A/m^2$).
Current density is in turn defined in terms of electric field $\vv{E}$ (units $V/m$)
\begin{equation}
    \vv{J} = \sigma \vv{E},
\end{equation}
where $\sigma$ is the specific electrical conductivity of intra-cellular fluid (typically 3.3 $S/m$).

The derivation of the cable equation is based on two assumptions:
\begin{enumerate}
    \item that charge disperion is effectively instantaneous for the purposes of dendritic modelling.
    \item that diffusion of magnetic field is instant, i.e. it behaves quasi-statically in the sense that it is determined by the electric field through the Maxwell equations.
\end{enumerate}
Under these conditions, $\vv{E}$ is conservative, and as such can be expressed in terms of a potential field
\begin{equation}
    \vv{E} = \nabla \phi,
\end{equation}
where the extra/intra-cellular potential field $\phi$ has units $mV$.

The derivation of the one-dimensional conservation equation \eq{eq:cable_balance} is based on the assumption that the intra-cellular potential (i.e. inside the cell) does not vary radially.
That is, potential is a function of the axial distance $x$ alone
\begin{equation}
    \vv{E} = \nabla \phi = \pder{V}{x}.
\end{equation}
This is not really true, because a potential field that is a variable of $x$ and $t$ alone can't support the axial gradients required to drive the potential difference over the cell membrane.
