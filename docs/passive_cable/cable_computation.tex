\documentclass[parskip=half]{scrartcl}

\usepackage{amsmath}
\usepackage{amssymb}
\usepackage[citestyle=authoryear-icomp,bibstyle=authoryear]{biblatex}
\usepackage{booktabs}
\usepackage[style=british]{csquotes}
\usepackage{fontspec}
\usepackage{siunitx}

\setmainfont[Numbers=Lowercase]{TeX Gyre Pagella}
\newfontfamily\dispfamily{TeX Gyre Pagella}
\addtokomafont{disposition}{\dispfamily}

\MakeAutoQuote{«}{»}
\DeclareMathOperator{\Res}{Res}

\addbibresource{cable.bib}

\title{Analytic solution to passive cable model}
\author{Sam Yates}
\date{October 20, 2016}

\begin{document}

\maketitle

\section{Background}

The Rallpack suite \parencite{bhalla1992} is a set of three tests for cable-based neuronal
simulators, for validation and benchmarking. Validation data is provided with the
distribution of the suite, which at the time of writing can now be found within the
\textsc{Genesis} simulator simulator distribution.\footnote{see \url{http://genesis-sim.org}.}

The first two of the tests are based on passive cable models which admit analytic
solutions, and the Rallpack suite includes code that can generate the reference curves
for these models. The reference code, however, requires hand-tuning of the iterations;
the authors state in the internal documentation that
«the use of 10 terms gives at least six figure accuracy», but this is for the
fixed time increments used in their dataset and is not validated within the code
or by the supplied material.

For flexibility of testing and cross-checking of the Rallpack reference data,
it would be useful to have at hand a robust calculator of the passive cable
potential solution.

\section{The Rallpack 1 model}

The model comprises a cylindrical cable segment with the physical and electrical
properties as described in Table~\ref{tbl:rallpack1}. A current of \SI{0.1}{\nA}
is injected at the left end of the cable from $t=0$, and the initial potential
is the reversal potential.

\begin{table}[ht]
    \centering
    \begin{tabular}{lSl}
        \toprule
        Term & {Value} & Property\\
        \midrule
        $d$    & \SI{1.0}{\um}                 & cable diameter \\
        $L$    & \SI{1.0}{\mm}                 & cable length \\
        $R_A$  & \SI{1.0}{\ohm\m}              & bulk axial resistivity \\
        $R_M$  & \SI{4.0}{\ohm\m\squared}      & areal membrane resistivity \\
        $C_M$  & \SI{0.01}{\F\per\m\squared}   & areal membrane capacitance \\
        $E_M$  & \SI{-65.0}{\mV}               & membrane reversal potential \\
        \bottomrule
    \end{tabular}
    \caption{Cable properties for the Rallpack 1 model.}
    \label{tbl:rallpack1}
\end{table}

The potential on a constant-radius passive cable is governed by the PDE
\begin{equation}
    \lambda^2 \frac{\partial^2 v}{\partial x^2} =
    \tau\frac{\partial v}{\partial t} + v - E,
\end{equation}
where $E$ is the membrane reversal potential, and $\lambda$ and $\tau$ are the
electrotonic length and time constants for the cable. It is convenient
to consider the electrical properties of the cable per unit-length, in terms
of the linear axial resistivity $r$, the linear membrane capacitance $r$,
and the linear membrane capacitance $c$. These determine $\lambda$ and $\tau$ by
\begin{gather*}
    \lambda = \frac{1}{r\cdot g},\\
    \tau = r\cdot c.
\end{gather*}

With the model boundary conditions,
\begin{subequations}
    \begin{align}
        v(x, 0) &= E, \\
        \left.\frac{\partial v}{\partial x}\right\vert_{x=0} & = -Ir, \\
         \left.\frac{\partial v}{\partial x}\right\vert_{x=L} & = 0,
    \end{align}
\end{subequations}
where $I$ is the injected current and $L$ is the cable length.

The solution $v(x, t)$ can be expressed in terms of the solution $g(x, t; L)$
to a normalized version of the cable equation,
\begin{subequations}
    \begin{align}
        \label{eq:normcable}
        \frac{\partial^2 g}{\partial x^2} & =
        \frac{\partial g}{\partial t} + g,
        \\
        \label{eq:normcableinitial}
        g(x, 0) &= 0,
        \\
        \label{eq:normcableleft}
        \left.\frac{\partial g}{\partial x}\right\vert_{x=0} & = 1,
        \\
        \label{eq:normcableright}
        \left.\frac{\partial g}{\partial x}\right\vert_{x=L} & = 0
    \end{align}
\end{subequations}
by
\begin{equation}
    v(x, t)= E - Ir\lambda \cdot g(\frac{x}{\lambda}, \frac{t}{\tau};  \frac{L}{\lambda}).
\end{equation}

\section{Solution to the normalized cable equation}

Let $G(x, s)$ be the Laplace transform of $g(x, t)$ with respect to $t$.
If the inverse transform of $G$ exists, it must agree with $g$ almost everywhere
for $t>0$, or for all $t>0$ given the continuity of $g$.

From
\eqref{eq:normcable} and \eqref{eq:normcableinitial},
\begin{equation}
    \label{eq:lap}
    \frac{\partial^2 G}{\partial x^2} = sG + G.
\end{equation}
The boundary conditions \eqref{eq:normcableleft} and \eqref{eq:normcableright} give
\begin{align}
    \label{eq:lapleft}
    \frac{\partial G}{\partial x}(0, s) & = \frac{1}{s}
    \\
    \label{eq:lapright}
    \frac{\partial G}{\partial x}(L, s) & = 0.
\end{align}

Solutions to \eqref{eq:lap} must be of the form
\begin{equation}
    G(x, s) = A(s) e^{mx} +B(s) e^{-mx}
\end{equation}
where $m=\sqrt{1+s}$. From \eqref{eq:lapleft} and \eqref{eq:lapright},
\begin{align*}
    mA(s) - mB(s) & = \frac{1}{s}
    \\
    mA(s)e^{mL} -mB(s)e^{-mL} & = 0
\end{align*}
and thus
\begin{align*}
    A(s) & = \frac{1}{sm (1-e^{2mL})}
    \\
    B(s) & = \frac{e^{2mL}}{sm (1-e^{2mL})}.
\end{align*}

Consequently,
\begin{equation}
    \begin{aligned}
        G(x, s) &= \frac{1}{ms}\cdot\frac{e^{mx}+e^{2mL-mx}}{1-e^{2mL}}\\
                &= - \frac{1}{ms}\cdot\frac{\cosh m(L-x)}{\sinh mL}.
    \end{aligned}
\end{equation}

\subsection*{Inversion}

Sufficient conditions for the inverse transform of $G(x,s)$ to exist
and be representable in series form are as follows
\parencite[][Theorem 4]{churchill1937}:
\begin{enumerate}
\item
    $G(x, s)$ is analytic in some right half-plane $H$,
\item
    the singularities of $G(x, s)$ are all poles, and
\item for some $k>1$, $|s^k G(x, s)|$ is bounded in $H$ and
    on the circles $|s|=\rho_i$, where $\rho_i$ is an unbounded monotonically
    increasing sequence.
\end{enumerate}
If in addition all the poles $\sigma_j$ of $G$ are simple, then
the series expansion of the inverse transform is given by
\begin{equation}
    g(x,t) = \sum_j e^{\sigma_j t}\, \Res(G;\sigma_j),\quad t>0.
\end{equation}

$G(x,s)$ has poles when $s=0$, $m=0$, or $\sinh mL=0$.
In the following, let
\begin{gather*}
    a_k = k\pi /L\\
    m_k = a_k i\\
    s_k = m_k^2 -1 = -k^2\pi^2/L^2-1.
\end{gather*}
for $k\in\mathbb{Z}$.
The poles are then $0$, $-1$, and $s_k$ for $k\geq 1$.

There are no branch points
arising from $m=\sqrt{1+s}$, as letting $m=-\sqrt{1+s}$ leaves $G$ unchanged.

For $|s+1|>\epsilon$,
\begin{equation}
    \begin{aligned}
        |s^{3/2}G(x,s)|^2
            & \leq (1+\epsilon)^{-1} \left| \frac{\cosh m(L-x)}{\sinh mL} \right|^2
        \\
            & \leq (1+\epsilon)^{-1} (1+|\coth mL|)^2
    \label{eq:gbounds}
    \end{aligned}
\end{equation}
which is bounded in the half-plane $\Re(s) \geq  -1+\epsilon$. 
As $\coth u+iv$ is periodic in $v$, \eqref{eq:gbounds} is also bounded outside
the regions $\{s: m\in D_k\}$, where $D_k$ are non-overlapping $\delta$-neighborhoods of $m_k$
for some fixed small $\delta>0$. These are contained within disks of radius
$2k\pi\delta/L+\delta^2$ about $s_k$ for $k\geq 1$, which then are separated by
circles $|s|=\rho_k$ about the origin for some $\rho_k$ in $(|s_k|, |s_k+1|)$.

\textbf{Pole at $s=0$}.\\
Recalling $m=\sqrt{1+s}$, $m$ and $\sinh mL$ are non-zero in
a neighbourhood of $s=0$, and so the pole is simple and
\begin{equation}
    \begin{aligned}
        \Res(G; 0) & = - \frac{1}{m}\cdot\left.\frac{\cosh m(L-x)}{\sinh mL}\right|_{s=0}\\
                   & = - \frac{\cosh (L-x)}{\sinh L}.
    \end{aligned}
\end{equation}

\textbf{Pole at $s=-1$}.\\
Let $G(x,s)=f(x,s)/h(s)$, where
\begin{equation}
    \begin{aligned}
        f(x, s) &= -\frac{1}{s}\cosh m(L-x)\\
        h(s) &= m\sinh mL.
    \end{aligned}
    \label{eq:hf}
\end{equation}
Noting that $dm/ds = \frac{1}{2}m^{-1}$,
\begin{equation}
    \begin{aligned}
        h'(s) &= \frac{1}{2}m^{-1}\sinh mL + \frac{1}{2}L\cosh mL \\
              &= \frac{1}{2}L + \frac{1}{2}L + O(m^2) \quad(m\to 0) \\
              &= L + O(s+1) \quad(s\to -1).
    \label{eq:hprime}
    \end{aligned}
\end{equation}
The pole is therefore simple, and
\begin{equation}
    \Res(G; -1) = f(x, -1)/h'(-1) = -\frac{1}{L}.
\end{equation}

\textbf{Pole at $s=s_k$, $k \geq 1$}.\\
Taking $h$ and $f$ as above \eqref{eq:hf},
$m_k$ is non-zero for $k\geq 1$ and
\[
    h'(s_k) = \frac{1}{2}L\cosh m_kL.
\]
Consequently the pole is simple and
\begin{equation}
    \begin{aligned}
        \Res(G; s_k)
            & = f(x, s_k)/h'(s_k)\\
            & =  -\frac{2}{s_k L}\frac{\cosh m_k(L-x)}{\cosh m_kL} \\
            & =  -\frac{2}{s_k L}\frac{\cosh m_kL\cosh m_kx-\sinh m_kL\sinh m_kL}{\cosh m_kL} \\
            & =  -\frac{2}{s_k L}\cosh m_k x,
    \end{aligned}
\end{equation}
as $\sinh m_k=0$.

In terms of $a_k$,
\begin{equation}
    \Res(G; s_k) = \frac{2}{L}\cdot\frac{1}{1+a_k^2}\cdot\cos a_kx.
\end{equation}

The series exapnsion for $g(x, t)$ therefore is
\begin{equation}
    g(x, t) = -\frac{\cosh(L-x)}{\sinh L} + \frac{1}{L}e^{-t}\left\{
        1+2\sum_{k=1}^\infty \frac{e^{-ta_k^2}}{1+a_k^2}\cos a_k x\right\}.
    \label{eq:theg}
\end{equation}

\section{Computation of series}

As $t$ approaches zero, the series \eqref{eq:theg} converges increasingly slowly.
For computational purposes, an estimate on the series residual allows the determination
of stopping criteria for a given tolerance.

Let $g_n$ be the partial sum
\begin{equation}
    g_n(x, t) = -\frac{\cosh(L-x)}{\sinh L} + \frac{1}{L}e^{-t}\left\{
        1+2\sum_{k=1}^n \frac{e^{-ta_k^2}}{1+a_k^2}\cos a_k x\right\}.
\end{equation}
so that $g(x, t) =\lim_{n\to\infty} g_n(x,t)$. Let $\bar{g}_n = |g-g_n|$ be the
residual. The $a_k$ form an increasing sequence, so
\begin{equation}
    \begin{aligned}
        \bar{g}_n(x,t)
            & \leq \frac{2}{L}e^{-t}\sum_{n+1}^\infty\frac{e^{-ta_k^2}}{1+a_k^2}\\
            & \leq \frac{2}{L}e^{-t}\int_{a_n}^\infty \frac{e^{-tu^2}}{1+u^2}\,du\\
            & < \frac{2}{L}e^{-t}\int_{a_n}^\infty \frac{e^{-tu^2}}{u^2}\,du.
    \end{aligned}
    \label{eq:gbar}
\end{equation}
Substituting $u=\sqrt{v/t}$ gives the identity
\begin{equation}
    \int_{a}^\infty \frac{e^{-tu^2}}{u^2}\,du
    = \frac{1}{2}\sqrt{t}\int_{a^2t}^\infty e^{-tv} v^{\frac{-3}{2}}\,dv
    = \frac{1}{2}\sqrt{t}\,\Gamma(-\frac{1}{2},a^2 t),
\end{equation}
where $\Gamma(\alpha, z)$ is the upper incomplete gamma function.

For real $\alpha<1$ and $z>0$, \textcite[][Theorem 2.3]{borwein2009} give the upper bound
\[
    \Gamma(\alpha, z) \leq z^{\alpha-1} e^{-z}.
\]
Substituting into \eqref{eq:gbar} gives
\begin{equation}
    \begin{aligned}
        \bar{g}_n(x,t)
            & < \frac{1}{L}e^{-t}\sqrt{t}\,\Gamma(-\frac{1}{2},a_n^2 t) \\
            & \leq \frac{1}{L}e^{-t}\sqrt{t}\,(a_n^2 t)^{-\frac{3}{2}}e^{-a_n^2t} \\
            & = \frac{e^{-t(1+a_n^2)}}{L t a_n^3}.
    \end{aligned}
\end{equation}

\printbibliography
\end{document}
