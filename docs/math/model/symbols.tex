%-------------------------------------------------------------------------------
%\subsubsection{Balancing Units}
%-------------------------------------------------------------------------------
Ensuring that units are balanced and correct requires care.
Take the system of linear equations that arises from the finite volume discretisation, specified in \eq{eq:ode_linsys} and \eq{eq:rhs_linsys}
\begin{equation}
    \label{eq:linsys_FV}
    \left[
        \frac{\sigma_i \cmi}{\Delta t} + \sum_{j\in\mathcal{N}_i} {\alpha_{ij}}
    \right]
    V_i^{k+1} - \sum_{j\in\mathcal{N}_i} { \alpha_{ij} V_j^{k+1}}
        =
    \frac{\sigma_i \cmi}{\Delta t} V_i^k -  \sigma_i(i_m^{k} - i_e).
\end{equation}

The choice of units for a parameter, e.g. $\mu m^2$ or $m^2$ for the area $\sigma_{ij}$, introduces a constant of proportionality wherever it is used ($10^{-12}$ in the case of $\mu m^2 \rightarrow m^2$).
Wherever terms are added in \eq{eq:linsys_FV} the units must be checked, and constants of proportionality balanced.

First, appropriate units for each of the parameters and variables are chosen in~\tbl{tbl:units}.
We try to use the same units as NEURON, except for the specific membrane capacitance $c_m$, for which $F\cdot m^{-2}$ is used in place of $nF\cdot mm^{-2}$.

\begin{table}[hp!]
\begin{tabular}{lllr}
    \hline
    term                      &   units                 &  normalized units                         & NEURON \\
    \hline
    $t$                       &   $ms$                  &  $10^{-3} \cdot s$                        & yes    \\
    $V$                       &   $mV$                  &  $10^{-3} \cdot V$                        & yes    \\
    $a,~\Delta x$             &   $\mu m$               &  $10^{-6} \cdot m$                        & yes    \\
    $\sigma_{i},~\sigma_{ij}$ &   $\mu m^2$             &  $10^{-12} \cdot m^2$                     & yes    \\
    $c_m$                     &   $F\cdot m^{-2}$       &  $s\cdot A\cdot V^{-1}\cdot m^{-2}$       & no     \\
    $r_L$                     &   $\Omega\cdot cm$      &  $10^{-2} \cdot A^{-1}\cdot V\cdot m$     & yes    \\
    $\overline{g}$            &   $S\cdot cm^{-2}$      &  $10^{4} \cdot A\cdot V^{-1}\cdot m^{-2}$ & yes    \\
    $g_s$                     &   $\mu S$               &  $10^{-6} \cdot A\cdot V^{-1}$            & yes    \\
    $I_e$                     &   $nA$                  &  $10^{-9} \cdot A$                        & yes    \\
    \hline
\end{tabular}
\caption{The units chosen for parameters and variables in NEST MC. The NEURON column indicates whether the same units have been used as NEURON.}
\label{tbl:units}
\end{table}

\subsection{Left Hand Side}
First, we calculate the units for the term $\frac{\sigma_i \cmi}{\Delta t}$, as follows
\begin{align}
    \left[ \frac{\sigma_i \cmi}{\Delta t} \right]
        &=
    \frac{10^{-12}\cdot m^2 \cdot s\cdot A\cdot V^{-1}\cdot m^{-2} } {10^{-3}\cdot s}
            \nonumber \\
        &=
    10^{-9} \cdot A \cdot V^{-1}. \label{eq:units_lhs_diag}
\end{align}
The units of $\alpha_{i,j}$ are:
\begin{align}
    \left[ \alpha_{i,j} \right]
        &=
    \left[ \frac{\sigma_{ij}}{ r_L \Delta x_{ij}} \right] \nonumber \\
        &=
    \frac{10^{-12}\cdot m^2}{ 10^{-2} \cdot A^{-1}\cdot V\cdot m \cdot 10^{-6}\cdot m}
        \nonumber \\
        &=
    10^{-4}\cdot A \cdot V^{-1}, \label{eq:units_lhs_lu}
\end{align}
which can be scaled by $10^{5}$ to have the same scale as in \eq{eq:units_lhs_diag}.

Thus, the RHS with scaling for units is:
\begin{equation}
    \label{eq:linsys_LHS_scaled}
    \left[
        \frac{\sigma_i \cmi}{\Delta t} + \sum_{j\in\mathcal{N}_i} {10^5\cdot\alpha_{ij}}
    \right]
    V_i^{k+1} - \sum_{j\in\mathcal{N}_i} { 10^5\cdot\alpha_{ij} V_j^{k+1}}.
\end{equation}
The implementation folds the $10^5$ factor into the $\alpha_{ij}$ terms when they are used to calculate the invariant component of the matrix during the initialization phase.

After this scaling, the units of the LHS are
\begin{align}
    \text{units on LHS}
        &= (10^{-9} \cdot A \cdot V^{-1})( 10^{-3} \cdot V) \nonumber \\
        &= 10^{-12} \cdot A. \label{eq:balanced_units}
\end{align}

\subsection{Right Hand Side}
The first term on the RHS has the same scaling factor as the LHS, so does not need to be changed.

Density channels and point processes describe the membrane current differently in NMODL;
as current densities ($10\cdot A\cdot m^{-2}$) and currents ($10^{-9}\cdot A$) respectively.
The current term can be written as follows:
\begin{equation}
    \sigma_i (i_m - i_e) \equiv \sigma_i \bar{i}_m + I_m - I_e,
\end{equation}
where $\bar{i}_m$ is the current density contribution from ion channels, $I_m$ is the current contribution from synapses, and $I_e$ is current contribution from electrodes.

The units of the current density as calculated via NMODL are
\begin{equation}
    \label{eq:im_unit}
    \unit{ \bar{i}_m } =  \unit{ \overline{g} } \unit{ V }
                       =  10 \cdot A \cdot m^{-2},
\end{equation}
so the units of the current from density channels are
\begin{equation}
    \unit{\sigma_i \bar{i}_m} = 10^{-12}\cdot{m}^2 \cdot 10 \cdot A \cdot m^{-2} = 10^{-11}\cdot A.
\end{equation}
Hence, the $\sigma_i \bar{i}_m$ term must be scaled by 10 to match units.

Likewise the units of synapse and electrode current
\begin{equation}
    \label{eq:Im_unit}
    \unit{ I_e } = \unit{ I_m } = \unit{ g_s } \unit{ V }
                 = 10^{-9}\cdot A,
\end{equation}
which must be scaled by $10^3$ to match units.

The properly scaled RHS is
\begin{equation}
    \label{eq:linsys_RHS_scaled}
    \frac{\sigma_i \cmi}{\Delta t} V_i^k -
        (10\cdot\sigma_i \bar{i}_m + 10^3(I_m - I_e)).
\end{equation}

\subsection{Putting It Together}
Hey ho, let's go: from \eq{eq:linsys_LHS_scaled} and \eq{eq:linsys_RHS_scaled} the full scaled linear system is
\begin{align}
    &
    \left[
        \frac{\sigma_i \cmi}{\Delta t} + \sum_{j\in\mathcal{N}_i} {10^5\cdot\alpha_{ij}}
    \right]
    V_i^{k+1} - \sum_{j\in\mathcal{N}_i} { 10^5\cdot\alpha_{ij} V_j^{k+1}} \nonumber \\
       & =
    \frac{\sigma_i \cmi}{\Delta t} V_i^k -
        (10\cdot\sigma_i \bar{i}_m + 10^3(I_m - I_e)).
\end{align}
This can be expressed more generally in terms of weights
\begin{align}
    &
    \left[
        g_i + \sum_{j\in\mathcal{N}_i} {g_{ij}}
    \right]
    V_i^{k+1} - \sum_{j\in\mathcal{N}_i} { g_{ij} V_j^{k+1}} \nonumber \\
       & =
    g_i V_i^k -
        (w_i^d \bar{i}_m + w_i^p(I_m - I_e)),
\end{align}
which can be expressed more compactly as
\begin{equation}
    Gv=i,
\end{equation}
where $G\in\mathbb{R}^{n\times n}$ is the conductance matrix, and $v, i \in \mathbb{R}^{n}$ are voltage and current vectors respectively.

In NestMC the weights are chosen such that the conductance has units $\mu S$, voltage has units $mV$ and current has units $nA$.

    \begin{center}

    \begin{tabular}{llll}
        \hline
        weight & value & units  & SI \\
        \hline
        $g_i$    & $10^{-3}\frac{\sigma_i \cmi}{\Delta t}$ & $\mu S$ & $10^{-6} \cdot A\cdot V^{-1}$ \\
        $g_{ij}$ & $10^2\alpha_{ij}$                       & $\mu S$ & $10^{-6} \cdot A\cdot V^{-1}$ \\
        $w_i^d$  & $10^{-2}\cdot\sigma_i$                  & $10^2\mu m^{-2}$ & $10^{-10}m^2$ \\
        $w_i^p$  & $1$                                     & $1$     & $1$ \\
        \hline
    \end{tabular}

    \end{center}
%------------------------------------------
\subsection{Supplementary Unit Information}
%------------------------------------------
Here is some information about units scraped from Wikipedia for convenience.

\begin{table*}[htp!]
    \begin{center}

    \begin{tabular}{llll}
        \hline
        quality & symbol & unit  & notes \\
        \hline
        energy     & $J$ & joule   $j$  & work to push 1 $N$ through 1 $m$ \\
        charge     & $q$ & coulomb $C$  & $6.25\cdot10^{18}$ electrons, $[A\cdot s]$ \\
        current    & $I$ & ampere  $A$  & $[C\cdot s^{-1}]$, $A$ is SI base unit\\
        voltage    & $V$ & volt    $V$  & potential work per unit charge \\
        resistance & $R$ & ohm $\Omega$ & recall Ohm's law $V=IR$ \\
        capacitance& $C$ & farad   $F$  & $C=\frac{q}{V}$, $[J\cdot C^{2}]$\\
        conductance& $g$ & siemens $S$  & $g=1/R$ i.e. inverse of resistance \\
        \hline
    \end{tabular}

    \vspace{20pt}

    \begin{tabular}{llll}
        \hline
        symbol & unit & equivalents & SI base \\
        \hline
        $J$    & $j$      &  $J\cdot s^{-1}$, $V\cdot A$ &
            $kg\cdot m^{2}\cdot s^{-2}$ \\

        $q$    & $C$      & $s\cdot A$ &
            $s\cdot A$ \\

        $I$    & $A$  & $C\cdot s^{-1}$ &
            $A$ \\

        $V$    & $V$      & $W\cdot A$ &
            $kg\cdot m^{2}\cdot s^{-3}\cdot A^{-1}$ \\

        $R$    & $\Omega$ & $V\cdot A^{-1}$ &
            $kg\cdot m^{2}\cdot s^{-3}\cdot A^{-2}$ \\

        $C$    & $F$      & $C\cdot V^{-1}$  &
            $kg^{-1}\cdot m^{-2}\cdot s^{4}\cdot A^{2}$ \\
        $g$    & $S$      & $A\cdot V^{-1}$  &
            $kg^{-1}\cdot m^{-2}\cdot s^3\cdot A^2$ \\
        \hline
    \end{tabular}

    \end{center}
    \caption{Symbols and quantities.}
\end{table*}

