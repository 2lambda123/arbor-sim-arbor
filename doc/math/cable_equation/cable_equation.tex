\documentclass[parskip=half]{scrartcl}

\usepackage{amsmath}
\usepackage{amssymb}
\usepackage[citestyle=authoryear-icomp,bibstyle=authoryear]{biblatex}
\usepackage{booktabs}
\usepackage[style=british]{csquotes}
\usepackage{fontspec}
\usepackage{mathtools}
\usepackage{ntheorem}
\usepackage{siunitx}

\usepackage{tikz}
\usetikzlibrary{angles,calc,intersections,quotes}

\usepackage[p,osf]{cochineal}
\usepackage[cochineal,vvarbb]{newtxmath}
\usepackage[cal=boondoxupr,calscaled=1.0]{mathalfa}

\newfontfamily\dispfamily{Cabin}[Scale=0.95]
\addtokomafont{disposition}{\dispfamily}

\MakeAutoQuote{«}{»}
\DeclareMathOperator{\Res}{Res}

\addbibresource{cable.bib}

\newcommand{\Int}[2]{\int_{#1}^{#2}\!}
\newcommand{\D}{\mathop{}\!d}

\newcommand{\cZ}{\mathcal{Z}}
\newcommand{\cC}{\mathcal{C}}

\theoremstyle{nonumberplain}
\newtheorem{lemma*}{Lemma}
\newenvironment{proof}{\textbf{Proof.}}{\hfill$\square$}

\title{Analytic solutions to the cable equation}

\author{Sam Yates}
%\date{July 29, 2019}

\sisetup{mode=text}

\begin{document}

% Work-around for siunitx vs newtxmath bug
\ExplSyntaxOn
    \cs_undefine:N \c__siunitx_minus_tl
    \tl_const:Nn \c__siunitx_minus_tl { - }
\ExplSyntaxOff

\maketitle

\section{The cable equation}

The cable equation describes the evolution of the potential
on a long, thin, conducting cable in a conducting medium, separated
by a leaky dielectric. It assumes that the behaviour
can be modelled entirely as a one dimensional problem,
in terms of the linear conductivity of the cable $\sigma$,
and the conductance $g$ and capacitance $c$ per unit length of the
dielectric. The potential $v(x, t)$ then satisfies
\begin{equation}
    (\sigma v')' = c \dot v + g v.
\end{equation}
where $f'$ denotes the derivative of $f$ with respect to the first
variable, and $\dot{f}$ the derivative with respect to the second.
The quantities $\sigma$, $g$, and $c$ are functions of position,
and will depend on the electrical and geometrical properties of
the system.

For a cable of radius $r(x)$ with constant bulk resistivity
$R_L$, areal capacitance $C_M$ and areal surface resistivity
$R_M$, these parameters are given by
\begin{align}
    \sigma(x) &= \pi r(x)^2 / R_L, \\
    g(x) &= 2 \pi r(x) \sqrt{1 + r'(x)^2} / R_M, \\
    c(x) &= 2 \pi r(x) \sqrt{1 + r'(x)^2} C_M.
\end{align}
Letting $\theta(x) = \arctan r'(x)$, the cable equation becomes
\begin{equation}
    \label{eq:constelec}
    \frac{R_M}{2 R_L}\left(
        2\sin\theta\cdot v' + r \cos\theta\cdot v''
    \right) =
    R_M C_M \dot v + v.
\end{equation}

\section{Analytic solutions for $v(x, t)$}

Two important special cases for the cable equation are the cylinder,
with $r(x)$ constant and $\theta(x)=0$, and the conical frustum,
with $\theta(x)$ constant and $r(x)=x\tan\theta$.

Given constant electrical properties, as in \eqref{eq:constelec},
a zero voltage at $t=0$, and Neumann boundary conditions, what
is $v(x, t)$? The Rallpack 1 model \autocite{bhalla1992} is
an example of such a passive cable model, and is discussed in
Appendix \ref{ap:rallpack}.

\subsection{The cylinder}

Suppose the cable lies on the interval $[0, B]$, with boundary
conditions $v'(0, t) = 0$ and $v'(B, t) = I$ and initial value
$v(x, 0) = 0$.

A change of variables gives
\begin{equation}
    v(x, t) = \lambda I\cdot u(x/\lambda, t/\tau),
\end{equation}
where
\begin{equation}
    \lambda = \sqrt{\frac{R_M r}{2 R_L}}, \quad \tau = R_M C_M, \quad b = B/\lambda,
\end{equation}
and $u$ satisfies
\begin{equation}
    \begin{aligned}
        u''(x, t) &= \dot u(x, t) + u(x, t),\\
        u(x, 0) &= 0,\\
        u'(0, t) &= 0,\\
        u'(b, t) &= 1.
    \end{aligned}
    \label{eq:ucyl}
\end{equation}

The corresponding time-invariant problem has solution $\hat u(x)$ where
\begin{equation}
    \hat u'' = \hat u,\quad \hat u'(0) = 0,\quad \hat u'(b) = 1,
\end{equation}
giving
\begin{equation}
    \hat u(x) = \frac{\cosh x}{\sinh b}.
\end{equation}

Consider separable transient solutions $u(x,t)-\hat u(x)$, with zero derivative
at $x=0$ and $x=b$, of the form $\psi(x)\eta(t)$. From \eqref{eq:ucyl},
$\eta(t)=e^{-\lambda t}$ for some $\lambda$ and $\psi$ satisfies
\begin{gather}
    \label{eq:efncyl}
    \psi'' + (\lambda-1)\psi = 0,\\
    \label{eq:efncylbc}
    \quad \psi'(0)=\psi'(b)=0.
\end{gather}
This is a regular Sturm-Liouville problem
\autocite[see e.g.][Theorem~2.1, p.~146]{reid1980} with discrete non-negative eigenvalues
$\lambda_k-1$ and corresponding orthogonal eigenfunctions $\psi_k$,
\begin{equation}
    \langle \psi_k, \psi_{k'} \rangle =
    \Int{0}{b} \psi_k(x)\psi_{k'}(x) \D x =0\quad\text{if $k\neq k'$}.
\end{equation}

The least eigenvalue corresponds to $\lambda_0 = 1$ with eigenfunction
$\psi_0(x)=1$.

For $\lambda>1$, \eqref{eq:efncyl} gives
$\psi(x) = c_1\cos\sqrt{\lambda-1}x + c_2\sin\sqrt{\lambda-1}x$. From $\psi'(0)=0$,
we have $c_2=0$. From $\psi'(b)=1$, it follows $c_1=0$ or $\sqrt{\lambda-1}b = k\pi$
for some positive integer $k$. Consequently we have eigenvalues and eigenfunctions,
\begin{gather}
    \label{eq:lkcyl}
    \lambda_k = 1+\left(\frac{k\pi}{b}\right)^{\mathrlap{2}},\\
    \psi_k(x) = \cos \frac{k\pi x}{b}.
\end{gather}
for integers $k\geq 0$, including the constant $\lambda=1$ case.

The time-dependent solution $u(x, t)$ then is
\begin{equation}
    u(x,t) = \hat u(x) - \sum_{k=0}^{\infty} a_k e^{-\lambda_k t}\psi_k(x),
\end{equation}
with $a_k$ determined by the initial conditions,
\begin{equation}
    0 = \langle u(x,0), \psi_k(x) \rangle
    = \langle \hat u(x), \psi_k(x) \rangle
    - a_k\langle \psi_k(x), \psi_k(x)\rangle,
\end{equation}
with
\begin{align}
    \langle \hat u(x), \psi_k(x) \rangle
    = \frac{1}{\sinh b}\Int{0}{b}\cosh x\cos\frac{k\pi}{b}x \D x
    = (-1)^k\lambda_k^{-1},
    \intertext{and}
    \langle \psi_k(x), \psi_k(x) \rangle
    = \Int{0}{b}\cos^2 \frac{k\pi}{b}x \D x
    =
    \begin{cases}
        b & \text{if $k=0$,}\\
        \frac{b}{2} & \text{otherwise.}\\
    \end{cases}
\end{align}
Consequently,
\begin{equation}
    \label{eq:cylseries}
    u(x,t) = \frac{\cosh x}{\sinh b} - \frac{1}{b}e^{-t} -
    \frac{2}{b}\sum_{k=1}^{\infty} (-1)^k \lambda_k^{-1} e^{-\lambda_k t} \cos\frac{k\pi x}{b},
\end{equation}
with $\lambda_k$ given in \eqref{eq:lkcyl}. A discussion on the residuals in this series can
be found in Appendix~\ref{ap:cylcomp}.

Note that the time derivative $\dot u(x,t)$ can be expressed in terms of Jacobi theta functions:
\begin{equation}
    \begin{aligned}
        \dot u(x,t) &= \frac{1}{b}e^{-t}\left(1+2\sum_{k=1}^{\infty} (-1)^ke^{-\frac{\pi^2 k^2 t}{b}}\cos\frac{k\pi x}{b}\right)\\
        &= \frac{1}{b}e^{-t}\theta_4\left(\frac{\pi x}{2 b}\middle| \frac{i\pi t}{b^2}\right).
    \end{aligned}
\end{equation}

\subsection{The tapered cable}

Consider the cable as a conical frustum on the interval $[A, B]$ with radius
$r(x) = x \tan\theta$ and boundary conditions $v'(A, t) = 0$, $v'(B, t) = I$,
and $v(x, 0) = 0$ (see Figure \ref{fig:tapered}). Reparameterizing gives
\begin{equation}
    v(x, t) = \lambda I\cdot u(x/\lambda, t/\tau),
\end{equation}
where
\begin{equation}
    \lambda = \frac{R_M\sin\theta}{2 R_L}, \quad \tau = R_M C_M, \quad a = A/\lambda, \quad b = B/\lambda,
\end{equation}
and $u$ satisfies
\begin{equation}
    \label{eq:conu}
    \begin{aligned}
        x u''(x, t) + 2 u'(x, t) &= \dot u(x, t) + u(x, t),\\
        u(x, 0) &= 0,\\
        u'(a, t) &= 0,\\
        u'(b, t) &= 1.
    \end{aligned}
\end{equation}

\begin{figure}[tbh]
    \begin{tikzpicture}[scale=0.8]
        \coordinate (origin) at (0,0);
        \coordinate (b0) at (15,0);
        \coordinate (b1) at ($ (b0) + (0,1.8) $);
        \coordinate (a0) at (5,0);

        \path [name path=tangent] (origin) -- (b1);
        \path [name path=avert] (a0) -- +(0,2);
        \path [name intersections={of=tangent and avert, by=a1}];

        \coordinate (a-1) at ($ (a1)!2!(a0) $);
        \coordinate (b-1) at ($ (b1)!2!(b0) $);

        \draw [name path=axis, dashed] (origin) -- (b0);

        \draw let \p1 = (a1), \n2 = {0.2*\y1} in
            (a1) arc (90:270:\n2 and \y1);

        \draw let \p1 = (a1), \n2 = {0.2*\y1} in
            [densely dotted] (a-1) arc (-90:90:\n2 and \y1);

        \draw let \p1 = (b1), \n2 = {0.2*\y1} in
            (b0) ellipse (\n2 and \y1);

        \draw (a1) -- (b1) (a-1) -- (b-1);
        \draw [dotted] (origin) -- (a1) (origin) -- (a-1);

        \pic [draw, "$\theta$", angle radius = 15ex, angle eccentricity=1.2] {angle = b0--origin--b1};

        \coordinate (lright) at ($ (b1)!2!(b0) - (0,40pt) $);
        \coordinate (lleft) at ($ (origin) + (lright) - (b0) $);

        \path (lleft) node (x0label) { $x=0$ };
        \path (lright) node (xblabel) { $x=B =b\lambda $ };
        \path ($ (lleft)!(a0)!(lright) $) node (xalabel) { $x=A =a\lambda$ };

        \draw [dotted] ($ (a-1)-(0,2ex) $) -- ($ (xalabel)+(0,2ex) $);
        \draw [dotted] ($ (b-1)-(0,2ex) $) -- ($ (xblabel)+(0,2ex) $);
        \draw [dotted] ($ (origin)-(0,2ex) $) -- ($ (x0label)+(0,2ex) $);
    \end{tikzpicture}
    \caption{Tapered cable coordinates.}
    \label{fig:tapered}
\end{figure}

Proceeding similarly to the cylinder problem, the solution can be presented as a sum of
a time-invariant solution $\hat u$ and a series of transients that are solutions to
a Sturm--Liouville problem.

The time-invariant problem is
\begin{equation}
    \begin{gathered}
        x\hat u''(x) + 2\hat u'(x) - \hat u(x) = 0,\\
        \hat u'(a) = 0,\quad \hat u'(b) = 1.
    \end{gathered}
\end{equation}

Multiplying by $x$ gives a differential equation of the form
\begin{equation}
    \label{eq:conicmu}
    x^2 g''(x) + 2x g'(x) + \mu x g(x) = 0
\end{equation}
for some $\mu$. This has solutions in terms of Bessel functions.
If $\cC_\nu$ is a solution to the Bessel equation
\begin{equation}
    z^2 f''(z) + z f'(z) - (z^2 - \nu^2) f(z) = 0,
\end{equation}
then $g(x) = x^{-\frac{1}{2}}\cC_1(\alpha x^\frac{1}{2})$ satisfies
\begin{equation}
    \label{eq:besg}
    x^2 g''(x) + 2x g'(x) + \frac{\alpha^2}{4} x g(x) = 0.
\end{equation}
Similarly, if $\cZ_\nu$ is a solution to the modified Bessel equation
\begin{equation}
    z^2 f''(z) + z f'(z) - (z^2 + \nu^2) f(z) = 0,
\end{equation}
then $g(x) = x^{-\frac{1}{2}}\cZ_1(\alpha x^\frac{1}{2})$ satisfies
\begin{equation}
    \label{eq:modbesg}
    x^2 g''(x) + 2x g'(x) - \frac{\alpha^2}{4} x g(x) = 0.
\end{equation}

The time-invariant solution $\hat u(x)$ satisfies \eqref{eq:modbesg} with $\alpha=2$, and so
can be written as
\begin{equation}
    \hat u(x) = \frac{1}{\sqrt{x}} \left(c_1 I_1(2\sqrt{x}) - c_2 K_1(2\sqrt{x})\right),
\end{equation}
with derivative
\begin{equation}
    \hat u'(x) = \frac{1}{x} \left(c_1 I_2(2\sqrt{x}) + c_2 K_2(2\sqrt{x})\right).
\end{equation}
Solving for the boundary conditions $\hat u'(a)=0$, $\hat u'(b)=1$ then determines the coefficients
$c_1$ and $c_2$, giving
\begin{equation}
    \label{eq:conuhat}
    \hat u(x) =
    \frac{b}{\sqrt{x}}\cdot
    \frac{K_2(2\sqrt{a})I_1(2\sqrt{x}) + I_2(2\sqrt{a})K_1(2\sqrt{x})}
    {K_2(2\sqrt{a})I_2(2\sqrt{b})-I_2(2\sqrt{a})K_2(2\sqrt{b})}.
\end{equation}

Separable transient solutions $u(x,t)-\hat u(x)$ will be of the form $e^{-\lambda t}\psi(x)$,
where $\psi(x)$ satisfies
\begin{gather}
    \label{eq:conpsi1}
    x\psi''(x) + 2\psi'(x) + (\lambda-1)\psi(x) = 0\\
    \label{eq:conpsi2}
    \psi'(a) = \psi'(b) = 0.
\end{gather}
Multiplying \eqref{eq:conpsi1} by $x$ gives
\begin{equation}
    \label{eq:conpsils}
    x^2\psi''(x) + 2x\psi'(x) + (\lambda-1)x \psi(x) =
    \frac{d}{dx}(x^2\psi'(x)) + (\lambda-1)x \psi(x) = 0,
\end{equation}
which together with \eqref{eq:conpsi2} forms a regular Sturm--Liouville problem with
non-negative eigenvalues $\lambda_k-1$ and corresponding orthogonal eigenfunctions
$\psi_k(x)$ with respect to the weight function $\rho(x) = x$.

For $\lambda_0=1$, the eigenfunction is $\psi_0(x)=1$. For $\lambda_k>0$,
equation \eqref{eq:conpsils} is of the form \eqref{eq:besg}, with $\alpha=2\sqrt{\lambda_k-1}$.
Writing $\omega_k$ for $\sqrt{\lambda_k-1}$, we have
\begin{equation}
    \psi_k(x) = \frac{1}{\sqrt{x}} \left(c_1^{(k)} J_1(2\omega_k \sqrt{x})) + c_2^{(k)} Y_1(2\omega_k \sqrt{x})\right),
\end{equation}
with derivative
\begin{equation}
    \psi_k'(x) = - \frac{\omega_k}{x} \left(c_1^{(k)} J_2(2\omega_k\sqrt{x})) + c_2^{(k)} Y_2(2\omega_k\sqrt{x})\right).
\end{equation}

The boundary conditions $\psi_k'(a) = \psi_k'(b) = 0$ imply
\begin{gather}
    f_2\left(\sqrt{\tfrac{b}{a}}, 2\omega_k\sqrt{a}\right) = 0,
    \intertext{where $f_\nu$ is the cross-product Bessel function}
    f_\nu(q, x) = J_\nu(qx)Y_\nu(x)-J_\nu(x)Y_\nu(qx).
\end{gather}
The eigenvalues for $k>0$ are then $\lambda_k=1+\omega_k^2$ with
\begin{equation}
    \label{eq:conomegak}
    \omega_k =  \frac{\chi_{2, k}\left(\sqrt{\scriptstyle \frac{b}{a}}\right)}{2\sqrt{a}}.
\end{equation}
where $\chi_{\nu, k}(q)$ is the $k$th positive root of $f_\nu(q, x)=0$. These solutions
can be computed efficiently via a Newton--Raphson scheme \autocite{sorolla2013}.
$c_1^{(k)}$ and $c_2^{(k)}$ are then determined up to scale; for example, we can
set
\begin{equation}
    c_1^{(k)} = J_2(2\omega_k\sqrt{x})^{-1},\qquad
    c_2^{(k)} = -Y_2(2\omega_k\sqrt{x})^{-1},
\end{equation}
giving
\begin{equation}
    \label{eq:conpsik}
    \psi_k(x) =\frac{1}{\sqrt{x}}\left(
    \frac{J_1(2\omega_k\sqrt{x})}{J_2(2\omega_k\sqrt{a})}
    -
    \frac{Y_1(2\omega_k\sqrt{x})}{Y_2(2\omega_k\sqrt{a})}
    \right).
\end{equation}

The complete solution then will be
\begin{equation}
    \label{eq:conuxt}
    u(x,t) = \hat u(x) - \sum_{k=0}^{\infty} a_k e^{-\lambda t} \psi_k(x),
\end{equation}
where the $a_k$ will be determined by the initial conditions.

As the $\psi_k$ are
eigenfunctions of the Sturm-Liouville problem (\ref{eq:conpsils}, \ref{eq:conpsi2}), they
are orthogonal with respect to the inner product with weight function $\rho(x)=x$,
\begin{equation}
    \langle f, g \rangle = \Int{a}{b} xf(x)g(x) \D x.
\end{equation}
Taking the inner product of $\langle u(x,0), \psi_k(x)\rangle$ then gives for each $k$,
\begin{equation}
    a_k = \frac{\Int{a}{b} x \hat u(x)\psi_k(x) \D x}{\Int{a}{b} x \psi_k(x)^2 \D x}.
\end{equation}

For $k=0$,
\begin{equation}
    \begin{aligned}
        a_0 &= \frac{1}{b-a} \cdot \Int{a}{b} x \hat u(x) \D x\\
        &= \frac{1}{b-a} \cdot \left. x^2\hat u'(x)\right|_a^b\\
        &= \frac{b^2}{b-a}.
    \end{aligned}
\end{equation}
(See Appendix \ref{ap:conid} for the derivation of this and the following integrals.)

For $k>0$,
\begin{equation}
    \begin{aligned}
        \Int{a}{b} x \hat u(x)\psi_k(x) \D x
        &= \frac{1}{1+\omega_k^2} \cdot \left. (x^2\hat u'(x)\psi_k(x)-x^2\hat u(x)\psi'_k(x))\right|_a^b\\
        &= \lambda_k^{-1} b^2\psi_k(b),
    \end{aligned}
\end{equation}
and
\begin{equation}
    \begin{aligned}
        \Int{a}{b} x \psi_k(x)^2 \D x
        &= \left.
        x^2\psi_k(x)^2 + \frac{x^2}{\omega_k^2}\psi_k(x)\psi'_k(x)
        + \frac{x^3}{\omega_k^2}\psi'_k(x)^2
        \right|_a^b\\
        &= b^2\psi_k(b)^2 - a^2\psi_k(a)^2,
    \end{aligned}
\end{equation}
giving
\begin{equation}
    a_k = \frac{1}{\lambda_k}\cdot \frac{b^2\psi_k(b)}{b^2\psi_k(b)^2-a^2\psi_k(a)^2}.
\end{equation}

This gives the time-dependent solution
\begin{equation}
    u(x,t) = \hat u(x) - \frac{b^2}{b-a}e^{-t} -
    \sum_{k=1}^{\infty} \lambda_k^{-1} e^{-\lambda_k t}
    \frac{b^2\psi_k(b)\psi_k(x)}{b^2\psi_k(b)^2-a^2\psi_k(a)^2}
\end{equation}
where $\hat u$ is given by \eqref{eq:conuhat}, $\psi_k$ is given by \eqref{eq:conpsik},
and $\lambda_k = 1 + \omega_k^2$ is given by \eqref{eq:conomegak}.


\section{Approximating the gradient}

In a finite volume discretization there
will be a computation of the approximation to the gradient
$v'(x)$ as a linear combination of two (or potentially more)
values of the discrete approximation of the voltage for points
near $x$.

If the coefficients in this approximation are constant in
time, they cannot account for source terms or values of the
voltage outside the points in question. The most faithful
such approximation should then reproduce the exact gradient $v'(x)$
in the source-free steady state form of the cable equation,
\begin{equation}
    (\sigma v')' = 0.
\end{equation}

The voltage $v$ is then determined by its values at
any two distinct points $a$ and $b$,
\begin{equation}
    \frac{v(x) - v(a)}{v(b) - v(a)} =
    \frac{\displaystyle \Int{a}{x} \sigma(z)^{-1} \D z}
         {\displaystyle \Int{a}{b} \sigma(z)^{-1} \D z},
\end{equation}
and correspondingly, the gradient is
\begin{equation}
    \sigma(x) v'(x) =
    \frac{v(b) - v(a)}
         {\displaystyle\Int{a}{b} \sigma(z)^{-1} \D z}.
\end{equation}
If $x$ is a point of discontinuity in $\sigma$,
the flux $\sigma(x)v'(x)$ is nonetheless well defined.

For a finite volume approximation, however, the available
estimates may correspond instead to surface area-weighted means
over control volumes. Consider two adjacent control volumes
on $[a, m]$ and $[m, b]$, with mean voltages $\bar v_a$ and
$\bar v_b$ given in terms of a weight function $w(x)$:
\begin{align}
    \bar v_a &= w_a^{-1} \Int{a}{m} w(x)v(x) \D x,\\
    \bar v_b &= w_b^{-1} \Int{m}{b} w(x)v(x) \D x,
\end{align}
with normalizing constants $w_a$ and $w_b$. With $\sigma v' = \kappa$
constant,
\begin{equation}
    \begin{aligned}
        \bar v_a
        &= w_a^{-1} \Int{a}{m} w(x) \left( v(m) + \kappa \Int{m}{x} \sigma(y)^{-1} \D y \right) \D x \\
        &= v(m) + \kappa \cdot w_a^{-1} \Int{a}{m} w(x) \Int{m}{x} \sigma(y)^{-1} \D y \D x,
    \end{aligned}
\end{equation}
and similarly, 
\begin{equation}
    \begin{aligned}
        \bar v_b
        &= v(m) + \kappa \cdot w_b^{-1} \Int{m}{b} w(x) \Int{m}{x} \sigma(y)^{-1} \D y \D x,
    \end{aligned}
\end{equation}
Taking the difference then gives a solution for $\kappa$,
\begin{equation}
    \kappa = (\bar v_b - \bar v_a) \cdot \left[
        w_b^{-1}\!\Int{m}{b} w(x)\! \Int{m}{x} \sigma(y)^{-1} \D y \D x\,-\,
        w_a^{-1}\!\Int{a}{m} w(x)\! \Int{m}{x} \sigma(y)^{-1} \D y \D x
        \right]^{\mathrlap{-1}}.
\end{equation}

\newpage
\appendix
\section{The Rallpack 1 model}
\label{ap:rallpack}

The Rallpack suite \autocite{bhalla1992} is a set of three models for the
validation and benchmarking of cable-based neuron simulators. The first of
these comprises a cylindrical passive cable of length $L$ and diameter $d$,
with a constant current applied at $x=0$. The membrane resistive current
density is given by $J = R_M^{-1}(v-E)$, where $E$ is a fixed reversal potential.
The values of the electrial and geometric parameters are given in
Table~\ref{tbl:rallpack1}.

The membrane potential can then be expressed as
\begin{equation}
    v(x, t) = E - \lambda I \cdot u(\frac{b-x}{\lambda}, \frac{t}{\tau}),
\end{equation}
where
\begin{equation}
    \lambda = \sqrt{\frac{R_M d}{4 R_L}},\quad
    \tau = R_M C_M,\quad
    b = L/\lambda,
\end{equation}
and $u(x,t)$ is given by \eqref{eq:cylseries}.

\begin{table}[htb]
    \centering
    \begin{tabular}{lll}
        \toprule
	Parameter & {} & {Value} \\
        \midrule
	Cable diameter                    & $d$    & \SI{1.0}{\um} \\
	Cable length                      & $L$    & \SI{1.0}{\mm} \\
	Bulk (axial) resistivity          & $R_L$  & \SI{1.0}{\ohm\m} \\
	Membrane resistivity              & $R_M$  & \SI{4.0}{\ohm\m\squared} \\
	Membrane specific capacitance     & $C_M$  & \SI{0.01}{\F\per\m\squared} \\
	Membrane reversal potential       & $E_M$  & \SI{-65.0}{\mV} \\
	Injected current                  & $I$    & \SI{0.1}{\nA} \\
        \bottomrule
    \end{tabular}
    \caption{Rallpack 1 parameters}
    \label{tbl:rallpack1}
\end{table}


\section{Series computation for the voltage on the uniform cylinder}
\label{ap:cylcomp}

How many terms in the series \eqref{eq:cylseries} need to be computed for a given tolerance?
We obtain bounds on the remainders in the series as follows.

Let $u_n(x,t)$ be the partial sum up to $k=n-1$ in \eqref{eq:cylseries}, and $r_n(x,t) = u(x,t) - u_n(x,t)$
be the remainder,
\begin{equation}
    r_n(x,t) = -\frac{2}{b}\sum_{k=n}^{\infty} (-1)^k\lambda_k^{-1}e^{-\lambda_k t}\cos\frac{k\pi x}{b}.
\end{equation}
Then $|r_n(x,t)|\leq R_n(t)$ for all $x$, with
\begin{equation}
    R_n(t) = \frac{2}{b}\sum_{k=n}^{\infty} \lambda_k^{-1}e^{-\lambda_k t}\\
    = \frac{2t}{b}\sum_{k=n}^{\infty} l(k)^{-1}e^{-l(k)}\\
\end{equation}
where
\[
    l(z) = \left(1+\left(\frac{z\pi}{b}\right)^2\right)\cdot t.
\]
The summand is monotonically decreasing in $k$, so
\begin{equation}
    \label{eq:rnigamma}
    R_n(t)
    \leq \frac{2t}{b}\Int{n}{\infty} l(k)^{-1}e^{-l(k)}\D k
    = \frac{\sqrt{t}}{b} \Int{\lambda_n t}{\infty} e^{-z}z^{-\frac{3}{2}} \D z.
\end{equation}

For $a>0$, $s>0$, integrating by parts gives
$\displaystyle \Int{a}{\infty} e^{-z}z^{-s}\D z < e^{-a}a^{-s}$,
and so
\begin{equation}
    R_n(t) < \frac{1}{b t}\lambda_n^{-3/2}e^{-\lambda_n t}.
\end{equation}

\section{Identities and integrals for tapered cable eigenfunctions}
\label{ap:conid}

\newcommand{\Fp}{\smash{F^{\mathrlap{\prime}}}\mkern-2.0mu}
\newcommand{\Gp}{\smash{G^{\mathrlap{\prime}}}\mkern-2.0mu}

Let
\begin{equation}
    \begin{aligned}
        F_k(z) &= z^{-\frac{k}{2}}\cZ_k(2z^{1/2}),&
        G_k(z) &= z^{-\frac{k}{2}}\cC_k(2\omega z^{1/2}),
    \end{aligned}
\end{equation}
where $C_k$ and $Z_k$ are solutions to the Bessel and modified Bessel equations
respectively,
\begin{equation}
    \begin{aligned}
        \cZ_k(z) &= c_1 I_k(z) + c_2 e^{i\pi k} K_k(z),&
        \cC_k(z) &= c_3 J_k(z) + c_4 Y_k(z)
    \end{aligned}
\end{equation}
for some constants $c_1$, $c_2$, $c_3$, $c_4$.

$F_k$ and $G_k$ are solutions of the differential equations
\begin{gather}
    z f''(z) + (k+1) f'(z) - z = 0\\
    \intertext{and}
    z g''(z) + (k+1) g'(z) + \omega^2 z = 0
\end{gather}
respectively, following the identity (2) of \autocite[p.~512]{lommel1879}.

Applying the derivative and recurrence relations for Bessel functions
\autocite[\S 10.6, \S10.29]{nistdlmf} then gives the following identities,
\begin{align}
    \label{eq:fgrel}
    \Fp_k(z) &= F_{k+1}(z), & z \Fp_k(z) &= -k F_k(z) + F_{k-1}(z),\\
    \Gp_k(z) &= -\omega G_{k+1}(z), & z \Gp_k(z) &= -k G_k(z) + \omega G_{k-1}(z).
\end{align}
When $k$ is an integer, these imply
\begin{equation}
    \begin{aligned}
        F_{-k}(z) &= z^k F_k(z),\\
        G_{-k}(z) &= (-z)^k G_k(z),
    \end{aligned}
    \qquad \text{for $k\in\mathbb{Z}$}.
\end{equation}

In the following, derivatives will be respect to $z$, and the argument $z$ to
$F_k$ and $G_k$ will be omitted where it is unambiguous.

\subsection*{The integral $\Int{}{} z F_1 G_1 \D z$}

The relations \eqref{eq:fgrel} give the identities
\begin{equation}
    \begin{aligned}
        \label{eq:fgrel2}
        (z^kF_k)' &= z^{k-1}F_{k-1}\\
        (z^kG_k)' &= \omega z^{k-1}G_{k-1}.
    \end{aligned}
\end{equation}
Consequently
\begin{equation}
    \begin{aligned}
        (z^2 F_2 G_1)' &= z F_1 G_1 + z^2 F_2 \Gp_1 = z F_1 G_1 - \omega z^2 F_2 G_2 \\
        (z^2 F_1 G_2)' &= z F_1 G_1 + z^2 \Fp_1 G_2 = z F_1 G_1 + z^2 F_2 G_2.
    \end{aligned}
\end{equation}
Cancelling the $F_2G_2$ term gives
\begin{equation}
    (z^2 F_2 G_1 + \omega z^2 F_1 G_2)' =
    (1+\omega^2) z F_1 G_1,
\end{equation}
and so
\begin{equation}
    (1+\omega^2) \Int{}{} z F_1 G_1 dz = z^2 F_2 G_1 + \omega z^2 F_1 G_2 = z^2 \Fp_1 G_1 - z^2 F_1 \Gp_1.
\end{equation}

\subsection*{The integral $\Int{}{} z F_1 \D z$}

As observed in \eqref{eq:fgrel2}, $(z^2 F_2)' = z F_1$, and so
\begin{equation}
    \Int{}{} z F_1 dz = z^2 F_2 = z^2 \Fp_1.
\end{equation}

\subsection*{The integral $\Int{}{} z G_1^2 \D z$}

Following the technique of \autocite[p.533]{lommel1879},
\begin{equation}
    \begin{aligned}
        (z^a G_k G_{m+1})'
        &= a z^{a-1} G_k G_{m+1} + z^a(\Gp_k G_{m+1} + G_k \Gp_{m+1})\\
        &= a z^{a-1} G_k G_{m+1} - \omega z^a G_{k+1}G_{m+1} + z^{a-1} G_k(\omega G_m - (m+1) G_{m+1})\\
        &= (a-m-1) z^{a-1} G_k G_{m+1} + \omega z^{a-1} (zG_{k+1}G_{m+1} - G_k G_m).
    \end{aligned}
\end{equation}
The last term is symmetric in $k$ and $m$, so
\begin{equation}
    (z^a G_k G_{m+1} - z^a G_m G_{k+1})' =
    (a-m-1) z^{a-1} G_k G_{m+1} - (a-k-1) z^{a-1} G_m G_{k+1}.
\end{equation}
Letting $a=k+1$ then gives,
\begin{equation}
    (z^{k+1} G_k G_{m+1} - z^{k+1} G_{k+1} G_m)' = (k-m) z^k G_k G_{m+1},
\end{equation}
and in particular,
\begin{equation}
    \label{eq:intzg1sq}
    \Int{}{} z G_1^2 \D z = z^2 G_1^2 - z^2 G_0 G_2.
\end{equation}

The result \eqref{eq:intzg1sq} can be expressed in terms of just $G_1$ and $\Gp_1$ by
application of the recurrence relationships:
\begin{equation}
    \label{eq:intzg1sqbis}
    \Int{}{} z G_1^2 \D z = z^2 G_1^2 + \frac{z^2}{\omega^2} G_1 \Gp_1 + \frac{z^3}{\omega^2} G^{\prime 2}_1.
\end{equation}


\printbibliography
\end{document}
